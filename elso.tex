\documentclass[]{article}
%Ez itt a preambulum
\usepackage[utf8]{inputenc}
\usepackage[T1]{fontenc}
\usepackage{times}
\usepackage{blindtext}
\usepackage[margin=1.5in]{geometry}
\begin{document}
\textit{Szevasz} \textbf{világ} $\sqrt(2)$
 $$\sqrt(2)$$
 \'a \'e \'W
 \"o
 \H{o}
 \ss
 \~n
 \o
 \underline{aláhúzott}
\textbackslash
\begin{equation}
	a_1^2+a_2^2=a_3^2\label{eq:equation}
\end{equation}
\begin{eqnarray}
	a_1^2+a_2^2&=&a_3^2\\
	a_1^2+a_2^2&=&a_3^2
\end{eqnarray}
\iffalse
Ez a rész egy komment
és többsoros.
\fi
\begin{flushleft}
	\blindtext
\end{flushleft}
\begin{center}
	\blindtext
\end{center}
szó  szó                    szó
Új bekezdés \par
Ez is új bekezdés\\
Ez új sor \newline
Ez is új sor \linebreak
itt volt egy sortörés
\begin{itemize}
	\item Számozatlan
	\item Lista
\end{itemize}
\begin{enumerate}
	\item Számozott 
	\item Lista
	\item[69] Élem
\end{enumerate}
\begin{table}
	\caption{Táblázat címe}
	\begin{tabular}{|l||cr|}
		E&R&T\\ \hline
		EEE&RRR&TTT\\ \hline
	\end{tabular}\label{tab:table}
\end{table}
\'a^3
\end{document}
